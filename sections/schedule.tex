
The research schedule for three years is presented in Figure \ref{fig:timeSchedule}.
It comprises the six work packages previously discussed.
\begin{figure}[h]
  \centering
  \noindent\resizebox{\textwidth}{!}{
  \begin{tikzpicture}
 \begin{ganttchart}[
complete/.style={fill=gray!50},
incomplete/.style={fill=white},
 y unit title=0.5cm,
 y unit chart=0.7cm,
 vgrid,hgrid,
 title height=1,
 ]{1}{33}
%labels
\gantttitle[]{2019}{6}                 % title 2
\gantttitle[]{2021}{12}             % title 2
\gantttitle[]{2022}{12}             % title 2
\gantttitle[]{2023}{3} \\              
\gantttitle{Q3}{3}                    
\gantttitle{Q4}{3}                      % title 3
\gantttitle{Q1}{3}
\gantttitle{Q2}{3}
\gantttitle{Q3}{3}
\gantttitle{Q4}{3}
\gantttitle{Q1}{3}
\gantttitle{Q2}{3} 
\gantttitle{Q3}{3} 
\gantttitle{Q4}{3} 
\gantttitle{Q1}{3} \\

\ganttbar[progress=100,inline=false]{WP 1}{1}{9}\\

% \ganttgroup[inline=false]{Group 3}{13}{24} \\ 
\ganttbar[progress=30,inline=false]{WP 2}{7}{18} \\ 
\ganttbar[progress=0,inline=false]{WP 3}{12}{24}\\ 
\ganttbar[progress=0,inline=false]{WP 4}{18}{26} \\ 
% \ganttbar[progress=0,inline=false]{Thesis}{20}{27} \\ 
    \ganttbar[progress=0,inline=false, bar progress label node/.append style={below left= 10pt and 7pt}]{Thesis}{27}{33} \\ 
\end{ganttchart}
\end{tikzpicture}
}
	\caption{Research Plan for 3 years.} 
  \label{fig:timeSchedule}
\end{figure}
