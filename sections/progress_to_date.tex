

\subsection{WP 1: sequence similarity sketching}
The publication outlining the method developed for work package 1 will be published in the proceedings of RECOMB 2021~\cite{joudaki2020fast}. Furthermore, the reference code implementation of this method, along with the code for the experiments, is now publicly available~\footnote{The source code is available at https://github.com/ratschlab/Project2020-seq-tensor-sketching}. 

When experimented against synthetically generated sequences, \emph{Tensor Sketching} technique demonstrates a higher accuracy than the other assessed sketching methods, particularly when the mutation rate is high. Furthermore, sketches can be computed in a dynamic fashion, allowing us to compute sketches of all substrings of a sequences in linear time. This dynamic sketches are then leveraged to design a better sketch, termed \emph{Tensor Slide Sketch}, that shows the highest accuracy among the methods assessed. Finally, tensor sketching can be viewed as a framework to tackle a wide range of relevant bioinformatics problems, since it is straightforward to extend tensor sketching to different settings.

My contributions were developing and analysis of tensor sketching, tensor slide sketching, and design of the experimental setup, and writing the publication along with the co-authors. Furthermore, I implemented the first version of the tensor sketching in C++, which served the basis of the current reference code.


\subsection{WP 2: Phylogeny reconstruction}
In order to assess utility of the the sketching framework for phylogeny reconstruction, the distance matrix of virus genomes is currently approximated based on the sketches, and compared to the exact distances based on the edit distance, and the performance of method is compared against MASH. 

My contributions have been devising the ideas for sub-quadratic clustering, and coming up with the experimental setup and evaluation metrics for this task. Furthermore, I have added parallelism and other optimizations to the main implementations in order to reduce the running time of the experiments. 
